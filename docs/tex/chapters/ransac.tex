\chapter{Geometria epipolarna}

W wizji stereoskopowej (ang. stereo vision) istnieją dwie metody odtwarzania
trójwymiarowości sceny. Pierwsza, klasyczna metoda, polega na odtworzeniu
struktury sceny z dwóch (lub więcej) ujęć o znanych parametrach soczewki i
wielkości matrycy (ang. camera resectioning) oraz położeniu ogniska w
przestrzeni naszego modelu. Da nam to możliwość oszacowania umiejscowienia
dowolnego punktu w przestrzeni na płaszczyźnie zdjęcia lub dowolnego punktu w
przestrzeni na podstawie dwóch linii od obydwóch ognisk przez obie projekcie
tego punktu. Metoda ta nie sprawdza się w sytuacji gdy parametry kamery się
zmieniają, np. w wyniku poruszenia lub zmiany otoczenia.  Druga metoda
polegająca na budowie linii epipolarnych, bliżej odpowiada działaniu systemów
występujących w naturze. W nieskalibrowanym (nieznanym) układzie, na podstawie
korelacji punktowych pomiędzy obrazami, wyznacza się macierz fundamentalną
(ang. fundamental matrix), dzięki której możemy odtworzyć ujętą scenę w
przestrzeni 3D. 

Obie te metody opierają się na tzw. triangulacji, procesie wyznaczającym punkt
w przestrzeni trójwymiarowej, na podstawie dwóch lub więcej dwuwymiarowych
projekcji tego punktu na płaszczyźnie. Mając dany obraz $I$, punkt
$M=(^{W}X,$$^{W}Y,$$^{W}Z,$$1)^T$ \todo{usunąć spacje} (w układzie współrzędnych
$W$), musi leżeć na linii prostej, przechodzącej przez ognisko $O_C$ i obraz
danego punktu $m=(^{I}X,^{I}Y,1)^T$ na płaszczyźnie $I$. Szukany punkt $M$
znajduje się na przecięciu dwóch takich linii, otrzymanych z dwóch projekcji. 

\todo{współczynnik skali, obrót, perspektywa}

\begin{figure}[h!] \centering
  \includegraphics[width=1\textwidth]{images/epipolar_geometry.png}
  \caption{Przykładowa Geometria epipolarna. Schemat przedstawia dwie otworkowe
  kamery z ogniskami w punktach $O_C$ i $O_{C^\prime}$ widzące punkt $M$.
  Projekcje tego punktu na płaszczyznach $I$ i $I^\prime$ są oznaczone $m$ i
  $m^\prime$, epipola $e$ i $e^\prime$, a linie epopolarne $l_{m}$ i
  $l_m^\prime$. Źródło: \cite{fm_overview} } \end{figure}

\section{Wyznaczanie macierzy fundamentalnej}

Macierz fundamentalna $F$ jest macierzą $3 \times 3$, która określa relacje
pomiędzy dowolnymi punktami $m$ i $m^\prime$ w wizji stereoskopowej, zgodnie ze
wzorem: $$m^TFm^\prime = 0\,.$$ H.C. Longuet-Higgins w publikacji "A Computer
Algorithm for Reconstructing a Scene From Two Projections" \cite{eight_point}
opublikował tzw.  Eight-Point Algorithm do wyznaczania macierzy fundamentalnej,
na podstawie minimum ośmiu par punktów $m_i=(u,v,1)$ i
$m_i^\prime=(u^\prime,v^\prime,1)$.  Każda para, tworzy jedno równanie liniowe,
zawierające 9 niewiadomych, będących elementami szukanej macierzy fundamentalnej
$F$: $$uu^\prime F_{11} + uv^\prime F_{21} + uF_{31} + vu^\prime F_{12} +
vv^\prime F_{22} + vF_{32} + u^\prime F_{13} + v^\prime F_{23} + F_{33} =
0\,.$$ Układ takich równań w uproszczeniu daje jedno równanie: $$Af=0\,,$$
gdzie $f$ jest wektorem zwierającym 9 elementów macierzy $F$. Ponieważ $f$ jest
zdefiniowana dla dowolnego mnożnika, należy nałożyć dodatkowy warunek, że norma
$\|f\|$ jest równa 1. Z tego wynika, że do rozwiązania tego układu równań
wystarczy osiem par punktów.

Zaimplementowany przeze mnie algorytm rozwiązywania układów równań to
uproszczona metoda Jacobi-ego dla macierzy symetrycznych (gdyż macierz $A$
zawsze jest symetryczna).

Ostatecznym rozwiązaniem jest wektor własny, odpowiadający najmniejszej
wartości własnej.  Wartości tego dziewięcioelementowego wektora tworzą macierz
fundamentalną o rozmiarze $3 \times 3$.

\section{Budowa linii epipolarnych}

Algorytm Random Sample Consensus (RANSAC) \cite{ransac} jest estymatorem,
którego celem jest dopasowanie odpowiedniego modelu do niepewnych
(doświadczalnych) danych wejściowych, oznaczonych pewnym szumem lub błędem.
Jednym z zastosowań tego algorytmu jest budowa geometrii epopolarnej.
Publikacja "Enhancing RANSAC by generalized model optimization"
\cite{loransac} pokazuje, że jest możliwe odnajdywanie linii
epipolarnych na podstawie jedynie trzech par korespondujących regionów, ale
tylko w połączeniu z takimi metodami parującymi, które dopasowują całe
kształty, z dokładnością do ich obrotu. Dzięki temu będziemy mogli uzyskać trzy
pary punktowych korespondencji pomiędzy obrazami, dla każdej pary regionów.
Szacunkowy model geometrii epipolarnej uzyskujemy losując trzy pary regionów
korespondujących, z których otrzymamy dziewięć par punktów niezbędnych do
obliczenia macierzy fundamentalnej $F$. Kolejnym krokiem jest przekształcenie
wszystkich sparowanych punktów z obrazów $I$ i $I^\prime$ macierzą $F$.
Algorytm LO-RANSAC \cite{loransac} (ulepszony RANSAC) powtarzamy dla obu
obrazów (na wejściu podając otrzymany zbiór $\mathcal{U}$ transformowanych
punktów), który przebiega następująco: \begin{enumerate} \item Losuj $m$ ze
    wszystkich $N$ punktów w zbiorze $\mathcal{U}$ (gdzie $m$ jest minimalną
    próbą niezbędną do określenia szukanego modelu; dla linii, $m$ jest równe
    2). \item Zbuduj model na podstawie wylosowanych danych. Linię definiuje
    równanie: $y - y_1 = (( y_2 - y_1 )\div( x_2 - x_1 )) (x - x_1)$.  \item
    Sprawdź jak dużo punktów nie odbiega od modelu o więcej niż progowy
    parametr $\Theta$. Punkty leżące w linii (ang. inliners) oznaczamy
    $\mathcal{I}$, a $I = |\mathcal{I}|$.  \item Jeżeli jest to najlepszy do
    tej pory szacunek ($I_k > I_j$ dla każdego $j < k$), zapamiętaj go. W
    przeciwnym razie przejdź do punktu 6. \item Wylosuj stałą liczbę punktów ze
    zbioru $I_k$ i przeprowadź optymalizację stosując algorytm IRLS
    (iteratively reweighted least squares). \item Jeżeli prawdopodobieństwo
    $\eta = (1-P)^k$, znalezienia lepszego modelu w $k$-tej próbie nie jest
    mniejsze od wartości progowej $\eta_0$ będącej parametrem, wróć do punktu
    1.  \end{enumerate}

Przyjmuje się, że para regionów jest spójna z wyznaczonym modelem, jeżeli
wszystkie jej trzy punkty są spójne. Prawdopodobieństwo $P$ wynosi: $$P =
\frac{{I \choose m}}{{N \choose m}} = \prod_{j=0}^{m-1}\frac{I - j}{N-j} \,.$$


\section{Odtwarzanie przekształceń geometrycznych}


