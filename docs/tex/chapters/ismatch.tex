\chapter{Dopasowanie kształtów}

Lokalnie pokrewne fragmenty obrazu (ang. Local Affine Frames) łączone w pary,
służą do ustalania korespondencji pomiędzy dwoma obrazami. Sposób opisu tych
regionów wpływa na jakość wyników algorytmu dopasowującego. Ze względu na nasze
wymagania, będziemy potrzebowali takiego sposobu reprezentacji, aby w jak
najmniejszym stopniu wpływał on na zmianę skali oraz perspektywy obrazu. Z
pomiarów opublikowanych w publikacji "Shape Guided Maximally Stable Extremal
Region (MSER) Tracking" wynika, że kontury znacznie dokładniej niż elipsy
opisują badane regiony i jednocześnie nie wymagają dużych zasobów do obliczenia
wyznacznika podobieństwa. Najszybszym obecnie algorytmem do dopasowania
kształtów jest IS-Match przedstawionym przez M. Donoser et al. \cite{ismatch}.
\todo{jakie są inne korzyści tego rozwiązania}

\section{Opis konturu}

Algorytmy dopasowania kształtów możemy podzielić na dwie kategorie: globalne i
lokalne. Podejście globalne porównuje całości konturów i działa pod warunkiem
braku znaczących deformacji takich jak zgięcia, częściowe przysłonięcia itp.
Dopasowanie lokalne pozwala na dopasowanie fragmentów konturu, ale osiąga
gorsze rezultaty przy dopasowaniu całości. Algorytm IS-Match (ang. Integral
Shape Match) posiada cechy obu tych podejść. Tak jak przy dopasowaniu
globalnym, kształt reprezentuje za pomocą punktów próbkowanych z konturu (stała
odległość między punktami). Opis konturu składa się natomiast z kątów
siecznych, które zawierają w sobie zarówno lokalne jak i globalne cechy, dzięki
czemu możliwe jest dopasowanie różnych, przekształconych, zmodyfikowanych, a
nawet częściowych, wersji konturu.

Deskryptor jest oparty o kąty $\alpha_{ij}$, które określają układ pomiędzy
punktami konturu. Kąt $\alpha_{ij}$ znajduje się pomiędzy sieczną z
referencyjnego punktu $P_{i}$ do punktu $P_{j}$ oraz sieczną z punktu $P_{j}$
do punktu $P_{j-\Delta}$, gdzie $i$ oraz $j$ przyjmują wartości od $1$ do $N$
czyli liczby punktów w konturze, a $\Delta$ jest parametrem. Taki sposób opisu
pozwala pozostać obojętnym na zmianę skali czy rotacje, przy której żaden kąt
się nie zmienia.

Po obliczeniu wszystkich kątów otrzymujemy macierz o rozmiarze $N \times N$,
opisującą nasz kontur. Elementy na diagonali zawsze mają wartość zero, gdyż
odpowiadają kątowi, którego oba ramiona kończą się w tym samym punkcie. Cechy
lokalne konturu zawarte są bliżej diagonali, a globalne dalej. Nie ma znaczenia
od którego punktu zaczniemy iterację, ponieważ macierz jest cykliczna i możemy
ją przesuwać wzdłuż diagonali.

\todo{Przykłady konturów i ich deskryptorów}

\section{Odnajdywanie podobieństw}

Załóżmy, że chcemy porównać dwa kształty. Naszym celem będzie znalezienie
podobnych do siebie fragmentów konturów, reprezentowanych przez macierze
kątowe: $A_{1}$ o rozmiarze $M \times M$ i $A_{2}$ o rozmiarze $N \times N$,
gdzie $M \leq N$.  Matematycznie oznacza to znalezienie takiego bloku o
rozmiarze $r \times r$, zaczynającego się w punkcie $A1(s,s)$ (leżącego na
diagonali tej macierzy), który zwróci różnicę kątową $D_{\alpha}(s,m,r)$, z
blokiem $A_{2}(m,m)$, poniżej zadanej wartości progowej. 

% \begin{equation}
%   D_{\alpha}(s,m,r) = \frac{1}{r^2} \sum_{i=0}^{r-1} \sum_{j=1}^{r-1} [A_{1}(s+i, s+j) - A_{2}(m+i, m+j)]^2.
%   \label{eq:wzor1}
% \end{equation}

Ponieważ musimy rozważyć każdy, dowolnie duży blok, zaczynający się w dowolnym
miejscu na diagonali, metoda iteracyjnego odejmowania wszyskich kątów
wymagałaby zbyt dużo mocy obliczeniowej i nie jest stosowana.

Efektywniejszą metodą obliczenia wyznacznika podobieństwa dla wszystkich trzech
parametrów $s,m,r$ jest budowa $N$ macierzy różnicowych (ang. descriptor
difference matrices) $M_{D}^{n}$ o rozmiarze $M \times M$, które dopasowują
kolejne punkty konturu $A_{1}$ (parametr s):

\begin{equation}
  M_{D}^{n} = A_{1}(1: M, 1: M) - A_{2}(n: n+M-1, n:n_M-1).
\end{equation}

Jeżeli naszym celem jest uzyskanie tylko jednej miary podobieństwa, która
określi to jak dobrze można dopasować cały mniejszy kontur do tak samo dużego
fragmentu większego, to będzie nią średnia wartość wszystkich elementów
macierzy $M_{D}^{n}$.

W przeciwnym razie, dla każdej macierzy różnicowej, należy szukać iteracyjnie
optymalnych wartości parametrów $m$ i $r$. Ponieważ dla jednej wartości
parametru $s$ możemy uzyskać wiele pasujących fragmentów konturu, należy
zwrócić wszystkie z nich.

Na dalszym etapie parowania konturów, oprócz średniej wartości na macierzy
różnicowej (lub jej fragmentu), powinniśmy także uwzgędnić jej wielkość (im
pasujący fragment jest większy, tym lepiej).
>>>>>>> 100672a... [wip] + pierwsza część trzeciego rozdziału o dopasowaniu kształtów


\section{Algorytm}

Algorytm na wejściu przyjmuje dwa kontury i zwraca dopasowane częściowo kontury i wyznacznik podobieństwa.
